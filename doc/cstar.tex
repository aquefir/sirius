%&program=xelatex
%&encoding=UTF-8 Unicode

\documentclass[12pt,english]{article}
\usepackage[a4paper,bindingoffset=1.5cm,margin=1cm,footskip=0.75cm]{geometry}

\usepackage{graphicx}
\usepackage{amssymb}
\usepackage{blindtext}
\usepackage{scrextend}

\widowpenalties 1 10000
\raggedbottom

\setcounter{tocdepth}{3}

\usepackage[bookmarks=true, unicode=true, hidelinks=true,
linktoc=section]{hyperref}

\newlength{\tindent}
\setlength{\tindent}{\parindent}
\setlength{\parindent}{0pt}
\renewcommand{\indent}{\hspace*{\tindent}}

\setlength{\parskip}{1.25em}

\pagenumbering{arabic}

\usepackage{mathspec}
\usepackage{fontspec}
\setallmainfonts{Palatino}
\setmonofont{Source Code Pro}

\renewcommand{\texttt}[1]{\ttfamily{\small{#1}}\normalfont{}}

\newcommand{\hr}{\noindent\rule{185mm}{0.4pt}}

\begin{document}
\begin{center}

\vspace*{9cm}

\Large{\textbf{ARQADIUM, LLC}} \\*
\vspace*{0.25cm}
\Huge{\textbf{The C$\star$ Programming Language}}
\end{center}

\clearpage{}

\thispagestyle{empty}

\vspace*{3cm}

\begin{center}
\LARGE{\textbf{Attention}}
\end{center}

\vspace*{1.5cm}

This document contains proprietary information of \textbf{ARQADIUM, LLC}
and/or its licensed developers and are protected by national and international
copyright laws. They may not be disclosed to third parties or copied or
duplicated in any form, in whole or in part, without the prior written consent
of ARQADIUM, LLC.

\clearpage{}

\tableofcontents{}

\clearpage{}

\section{Introduction}
\label{s-introduction}

\subsection{Purpose}
\label{s-introduction-purpose}

C$\star$ is a systems programming language derived from ANSI C with the aim of
facilitating a system of ``Law and Order''. Like C, it is statically and
strongly typed. It takes after C's strengths in \textit{communicativity} and
lack of genericism. By providing ``Law and Order'', C$\star$ provides a better
way of achieving program correctness than current forms of contract
programming or test-driven development. In systems programming, correctness is
crucial, so C$\star$ is a great leap forward for the efficacy of the field.

\subsection{Document conventions}
\label{s-introduction-document_conventions}

\paragraph{Normative terminology}
This document makes use of ``normative terminology'' to communicate specific
meanings of terms which would otherwise be vague or confusing. Wherever these
appear, they will be highlighted or bolded.

\paragraph{Requirement levels}
The key words \textbf{must}, \textbf{must not}, \textbf{required},
\textbf{shall}, \textbf{shall not}, \textbf{should}, \textbf{should not},
\textbf{recommended}, \textbf{may}, \textbf{may not}, and \textbf{optional} in
this document are to be interpreted as described in RFC 2119.

\paragraph{Definition of general terms}
The following terms are used in this document:

\begin{labeling}{Unspecifiedd behaviour}
\item [\textbf{Program}] A collection of actions expressed in C$\star$ source
code
\item [\textbf{Machine}] A Turing machine to execute the \textbf{program}
\item [\textbf{Implementation}] The software that translates a
\textbf{program} into a form appropriate for the \textbf{machine}
\item [\textbf{Total system}] The portion of a \textbf{program} for which the
entire contents can be fully purveyed by the \textbf{implementation}
\item [\textbf{Object}] a region of data storage in a \textbf{machine}.
\textbf{Objects} are composed of contiguous sequences of bits,
the size, ordering and encoding of which are either explicitly specified or
\textbf{implementation}-defined
\item [\textbf{Type}] Notation to specify the specific \textit{nature}, such
as the size, properties, and possible forms of data of an \textbf{object}
\item [\textbf{Rune}] an \textbf{object} at least 21 bits in size,
representing a single Unicode code point
\item [\textbf{Constraint}] a specific limitation imposed on the possible form
of some kind of \textbf{objects}
\item [\textbf{Law}] An explicit, user-defined \textbf{constraint} to be
applied to specific \textbf{types} of \textbf{objects}
\item [\textbf{Pact}] A \textit{falsifiable} \textbf{constraint} to be checked
on certain \textbf{types} of \textbf{objects}
\item [\textbf{Marshal}] Validation boilerplate code that is dispatched to
handle falsified \textbf{pacts}
\item [\textbf{Address}] an \textbf{object} representing the location of an
\textbf{object}, which may imply granularity \textbf{constraints} to a nearest
number of bits (often 8)
\item [\textbf{Unspecified behaviour}] behaviour of a \textbf{program} for
which this specification defers to be \textbf{implementation}-defined
\item [\textbf{Undefined behaviour}] action(s) expressed in C$\star$ source
code for which this specification explicitly disclaims any and all claims of
behaviour
\end{labeling}

\subsection{Compliance}
\label{s-introduction-compliance}

A conforming \textbf{implementation shall} use only features of the language and library specified in this document. It \textbf{shall not} produce \textbf{programs} dependent on any \textbf{undefined behaviour}. The
\textbf{implementation shall not} exceed any minimum limit imposed upon it.

\section{The Language}
\label{s-language}

C$\star$ source code files are \textbf{recommended} to be given the file
extension ``\texttt{.cst}''.

\subsection{Compilation phases}
\label{s-language-compilation_phases}

The process called \textit{compilation} is divided into several sequential
and functionally independent \textit{phases.} The \textbf{implementation}
provides the functionality of compilation.

\begin{labeling}{source text decodee}
\item [\textbf{Source text decode}] Discern the source text's character set
encoding and transform it into the desired form by the \textbf{implementation}
\item [\textbf{Shebang}] If the first line of the source text begins with
``\texttt{\#!}'', then discard it
\item [\textbf{Lexing}] Source text is \textit{lexed} into an array of
tokens
\item [\textbf{Parsing}] The tokens are parsed to make an \textit{abstract
syntax tree,} or AST
\item [\textbf{Semantic analysis}] The AST is traversed to declare variables,
generate symbol tables, validate laws, and determine the meaning of the
\textbf{program}
\item [\textbf{Optimisation}] Optional re-traversal to create better
semantically-equivalent renditions of the \textbf{program}
\item [\textbf{Code generation}] The \textbf{program} semantics are used to
create the executable code for the \textbf{machine}
\end{labeling}

\subsection{Memory model}
\label{s-language-memory_model}

In C$\star$, the fundamental unit of memory is the \textit{bit.} Memory is
composed of contiguous stretches of bits, often with certain assumptions
permitted about its regularity and granularity of access. There are three (3)
kinds of memory:

\begin{labeling}{Constant memoryy}
\item [\textbf{Private memory}] Same as in OpenCL parlance; in CUDA terms it
may be called ``registers'' or ``local memory''; in general-purpose CPU terms
it is thread-local storage. Writable, and only accessible from a single
execution context.
\item [\textbf{Shared memory}] Same as in CUDA parlance; in OpenCL terms it is
called ``local memory''; in CPU terms it is typical, often heap-allocated
memory. Writable and accessible from potentially multiple concurrent execution
contexts, this is the only memory category that demands manual
synchronisation.
\item [\textbf{Constant memory}] Shared memory that is read-only for all
execution contexts. This memory can be shared and used without need for
synchronisation across multiple execution contexts, but is only modified at
the moment it was declared.
\end{labeling}

Multiple threads accessing \textbf{private memory} constitutes
\textbf{undefined behaviour}. Writing to \textbf{constant memory} at any time
after its declaration constitutes \textbf{undefined behaviour}.

In addition to the three kinds of memory, C$\star$'s memory model also
prescribes two (2) jurisdictions of memory:

\begin{labeling}{Foreign memoryy}
\item [\textbf{Native memory}] Memory for which its entire lifetime and all of
its access is confined to the \textbf{total system}. \textbf{Laws} can be
enacted upon \textbf{types} of \textbf{objects} residing in native memory.
Variable declarations default to this jurisdiction.
\item [\textbf{Foreign memory}] Memory that may be accessed, created and/or
destroyed outside of the \textbf{total system}. All \textbf{laws} enacted on a
\textbf{type} of \textbf{object} in foreign memory do not apply; instead, the
user must apply \textbf{pacts} and \textbf{marshalling} code to validate such
data. Use of either the \texttt{extern} or \texttt{volatile} storage modifiers
declares a variable as residing in foreign memory.
\end{labeling}

\subsection{Logic \& control flow}
\label{s-language-logic_control_flow}

C$\star$ provides functions as the vehicle for collating code. Braces provide
block scoping for variables. Declarations must be hoisted to the top of a
block, separated from the logic code, as they are in ANSI C. C$\star$ provides
the \texttt{if} and \texttt{else} keywords and a ternary syntax for expressing
conditional logic. The \texttt{do}, \texttt{for} and \texttt{while} keywords
enable loop construction, and the \texttt{break} and \texttt{continue}
keywords enable changing the control flow of loops. The \texttt{return}
keyword facilitates the termination of a function's execution, and optionally
the returning of one or more values to the function's caller. The language
provides several logical and bitwise operators:

\begin{labeling}{\textbf{\texttt{>>>=}}}
\item [\textbf{\texttt{\&\&}}] Short-circuit / logical AND.
\item [\textbf{\texttt{\&}}] Long-circuit / bitwise AND.
\item [\textbf{\texttt{||}}] Short-circuit / logical OR.
\item [\textbf{\texttt{|}}] Long-circuit / bitwise OR.
\item [\textbf{\texttt{\textasciicircum}}] Exclusive OR / XOR.
\item [\textbf{\texttt{!}}] Logical NOT.
\item [\textbf{\texttt{\textasciitilde}}] Bitwise NOT.
\item [\textbf{\texttt{<<}}] Logical bit-shift left.
\item [\textbf{\texttt{>>}}] Logical (unsigned) bit-shift right.
\item [\textbf{\texttt{>>>}}] Arithmetic (signed) bit-shift right.
\item [\textbf{\texttt{+}}] Arithmetic addition.
\item [\textbf{\texttt{-}}] Arithmetic subtraction.
\item [\textbf{\texttt{\**}}] Arithmetic multiplication.
\item [\textbf{\texttt{/}}] Arithmetic division.
\item [\textbf{\texttt{\%}}] Arithmetic modulus (remainder).
\end{labeling}

All of the above operators have assignment statement variants, except for the
logical NOT operator. Notably, this does include the short-circuit AND and OR
operators as well: \texttt{x \&\&= y;} is equivalent to \texttt{x = x \&\& y;}
so that \texttt{x} is only modified if it is already truthy, i.e. non-zero.

There are also several conditional operators in C$\star$:

\begin{labeling}{\textbf{\texttt{===}}}
\item [\textbf{\texttt{==}}] Equal to.
\item [\textbf{\texttt{!=}}] Not equal to.
\item [\textbf{\texttt{<}}] Less than.
\item [\textbf{\texttt{<=}}] Less than or equal to.
\item [\textbf{\texttt{>=}}] Greater than or equal to.
\item [\textbf{\texttt{>}}] Greater than.
\end{labeling}

As is the case with C, C$\star$ has no generic ``type system''; instead, all
types are ultimately numeric, so there is no such thing as type coercion and
also no need for any conditional operators that distinguish by their coercion
behaviour.

\subsection{Preprocessing \& headers}
\label{s-language-preprocessing_headers}

Like C, C$\star$ uses the C preprocessor, or CPP for short. Conforming
C$\star$ implementations \textbf{must} define the CPP macro constant named
\texttt{\_\_cstar} to value \texttt{1}. For C$\star$-specific headers, the file
extension \texttt{.hst} is \textbf{recommended}.

\subsection{The total system}
\label{s-language-total_system}

C$\star$ formalises the ``compilation unit'' as it is known in ANSI C into a
more structurally rigourous ``total system''. When the \textbf{implementation}
has complete analytical discretion over the constitution of a \textbf{program}
as it exists in a single compilation unit, it can enforce \textbf{laws} upon
all of the data types within that context.

This has several implications:

\begin{enumerate}
\item The only kind of linkage C$\star$ offers is what C would call ``external
linkage''; C's ``internal linkage'' is merely file scope with no linkage in
C$\star$
\item C$\star$'s provided storage modifiers become much more hierarchical and
straightforward, ending up tiered in increasing levels of externality
\item Since compilation units and files already have a strong correspondence,
the user of C$\star$ must either build them in one command invocation, or
\texttt{\#include} additional \texttt{.cst} files into one main source file as
a single unit. C$\star$ is not authoritarian – the \textbf{implementation}
does not care how systems are laid out, but merely that they \textit{are} laid
out
\end{enumerate}

\subsection{Storage classing}
\label{s-language-storage_classing}

C$\star$ provides only four (4) storage modifiers:

\begin{labeling}{\textbf{\texttt{registerr}}}
\item [\textbf{\texttt{const}}] Declares a variable to be immutable. It will
have static storage duration when declared at file scope, and automatic
storage duration when declared at block scope.
\item [\textbf{\texttt{extern}}] Declares a variable to require external
linkage, in the case its ultimate storage location is outside the total 
system. This always implies \texttt{volatile}.
\item [\textbf{\texttt{volatile}}] Declares a variable to be
\textit{volatile}, as in modifiable from outside of the total system. Where a
foreign module would identify a variable by \texttt{extern}, that variable
must be declared \texttt{volatile} in the system in which it resides.
\item [\textbf{\texttt{register}}] Declares a local variable with additional
semantic restrictions to allow more optimisation opportunities. This is only
allowed at block scope and in function parameter lists, not including forward
declarations. Variables qualified as \texttt{register} cannot have their
address taken, and \texttt{register}-qualified arrays cannot be converted to
pointers.
\end{labeling}

Having a \texttt{static} keyword in C$\star$ has proven redundant, as it is
unnecessary to declare variables as having static storage duration from block
scope, and linkage is presumed to be none unless the \texttt{volatile} or
\texttt{extern} keywords are used in which case it is external linkage anyway.
In C parlance, functions are always \texttt{extern}, but this does not affect
their semantics due to the mechanism of \textbf{marshalling}.

C11's \texttt{\_Thread\_local} storage modifier is also redundant, as C$\star$
provides better semantics for construing the level of isolation provided with
a variable than that. A sub-system can be created that is intended to run as
the kernel of potentially many threads, even heterogeneously on various kinds
of computers. From any system's perspective, it is just another system to
interface with.

Constant data cannot be made accessible outside of a system directly. Instead,
the solution is to provide a function as an interface for returning the data
instead.

\subsection{Laws}
\label{s-language-laws}

C$\star$'s most novel features are the ones of ``law \& order''; these include
the construct of \textbf{laws}, and the special kind of code at the beginning
of functions called \textbf{marshals}.

\textbf{Law}s can have a name, or be \textit{anonymous}, in the same syntax as
\texttt{struct}s and \texttt{enum}s. Before their body there is placed a colon
followed by a comma-separated list of one or more \textbf{types}; if the
\textbf{law} is named, this may be omitted. Observe:

\texttt{law no\_negs \\*
\{ \\*
\hspace*{1cm}\_ >= 0; \\*
\}; \\*
 \\*
law no\_negs : s8, s16, s32, s64; \\*
 \\*
law : u8, u16, u32, u64 \\*
\{ \\*
\hspace*{1cm}\_ != 0; \\*
\};}

All assignments and modifications to variables of the given types are
statically checked to ensure that it is impossible for the constraints of the
\textbf{laws} to be violated in any circumstances. For example, given a
decrementing \texttt{for} loop (shown below) and the \textbf{laws} above, it
must be communicated statically by the programmer that it is impossible for
the variable being enforced to ever have an illegal state. So, the following
code is illegal:

\texttt{s32 i; \\*
u8 blocks[32]; \\*
for(i = 16; blocks[i] != 0xFF; --i) \\*
\{ \\*
\hspace*{1cm}/\** ... \**/ \\*
\}}

This is illegal as it violates the named \textbf{law} ``\texttt{no\_negs}''.
It is not clear due to truncation for brevity whether the anonymous
\textbf{law} applying to \texttt{blocks} as a \texttt{u8[]} is violated as
there are no assignments or modifications to it shown.

\textbf{Laws} are assumed to have been violated if the variables they apply to
are used without due process. This is the case with the careless use of the
counter variable \texttt{i}, as it is not proactively checked to ensure it
never decrements below zero. In this trivial program, ensuring the value is
never negative ensures that the array access never exhibits \textbf{undefined
behaviour}, although in the real world with C$\star$ using the \texttt{ptri}
type would be the idiomatic solution.

If there were a \textbf{law} applying to \texttt{u8[]}s that mandated the
first element of all such types be equal to \texttt{0xFF}, then this code
would become legal, as it could be inferred from the conditional present in
the loop that the value of \texttt{i} never goes below zero.

\subsection{Type definitions}
\label{s-language-type_definitions}

C$\star$ has only one fundamental primitive type: the bit. This is construed
through the \texttt{bit} keyword. Additional primitives are defined as
register arrays of bits, and the \textbf{implementation} is expected to
optimise such arrays as appropriate for the \textbf{machine}. For example, a
register array of 25 bits may be returned through a register at least 32 bits
in size. A regular array of such a type will be bit-packed by default; more
optimal access can be achieved for a given \textbf{machine} by storing or
converting the elements into a more natural size for that \textbf{machine}.

\subsubsection{Provided primitives}
\label{s-language-type_definitions-provided_primitives}

\begin{labeling}{\textbf{\texttt{ptrii}}}
\item [\textbf{\texttt{s8}}] Signed 8-bit integer.
\item [\textbf{\texttt{s16}}] Signed 16-bit integer.
\item [\textbf{\texttt{s32}}] Signed 32-bit integer.
\item [\textbf{\texttt{s64}}] Signed 64-bit integer.
\item [\textbf{\texttt{s128}}] Signed 128-bit integer.
\item [\textbf{\texttt{u8}}] Unsigned 8-bit integer.
\item [\textbf{\texttt{u16}}] Unsigned 16-bit integer.
\item [\textbf{\texttt{u32}}] Unsigned 32-bit integer.
\item [\textbf{\texttt{u64}}] Unsigned 64-bit integer.
\item [\textbf{\texttt{u128}}] Unsigned 128-bit integer.
\item [\textbf{\texttt{f16}}] Half-precision floating-point number.
\item [\textbf{\texttt{f32}}] Single-precision floating-point number.
\item [\textbf{\texttt{f64}}] Double-precision floating-point number.
\item [\textbf{\texttt{f128}}] Quad-precision floating-point number.
\item [\textbf{\texttt{f256}}] Octo-precision floating-point number.
\item [\textbf{\texttt{offs}}] Signed integer sized the same as a pointer.
\item [\textbf{\texttt{ptri}}] Unsigned integer sized the same as a pointer.
\end{labeling}

\subsubsection{Types and ``law \& order''}
\label{s-language-type_definitions-types_law_order}

For the purposes of \textbf{law} \& order, type definitions are distinct from
one another even as they may be ``compatible types'' in C parlance. This is
needed to make it possible for \textbf{laws} to apply only to the intended
data types in use. Observe:

\texttt{
typedef u8 mybyte; \\*
 \\*
law : mybyte; \\*
\{ \\*
\hspace*{1cm}\_ != 0; \\*
\};}

With the above \textbf{law} in force, variables typed as \texttt{mybyte} must
always be non-zero. This does not apply to variables typed as \texttt{u8},
even though they are what C would call ``compatible types''. Since in this
situation \texttt{mybyte} is \textit{wholly more restrictive} than its base
type \texttt{u8}, it can be cast to \texttt{u8} without special analysis.

C$\star$'s \textbf{law} \& order is enforced analytically within the
\textbf{total system}, so that the need for casting in the other direction
goes away. In fact, explicit casting in any sense for the purposes of
\textbf{law} \& order is unnecessary. Effectively, this means that passing a
variable of type \texttt{u8} to a function that takes a \texttt{mybyte}
constitutes a fully retroactive invisible cast to all possible data paths of
that \texttt{u8} as if it were called a \texttt{mybyte}, all without any
explicit intervention or hinting from the programmer.

\subsection{Laws for structs}
\label{s-language-laws_structs}

\textbf{Laws} are applied to \texttt{struct}s in much the same way they are to
primitive types. Dot notation is used to access the members of the
\texttt{struct}, and this is done in the context of the \texttt{struct} as a
whole. Observe:

\texttt{struct bla \\*
\{ \\*
\hspace*{1cm}u32 a; \\*
\hspace*{1cm}s16 b; \\*
\hspace*{1cm}s16 c; \\*
\}; \\*
 \\*
law foo : struct bla \\*
\{ \\*
\hspace*{1cm}\_.a >= 10; \\*
\hspace*{1cm}\_.a <= 1000; \\*
\hspace*{1cm}\_.b >= -1; \\*
\hspace*{1cm}\_.c >= -1; \\*
\};}

The \textbf{laws} applied to \texttt{struct} members only apply to them as
part of their parent type. In other words, they are enacted on data types, but
those types are not affected outside that specific \texttt{struct}.

Arrow notation for accessing pointer members is not allowed. See the next
section for dealing with array and pointer types in \textbf{laws}.

\subsection{Laws for arrays and pointer types}
\label{s-language-laws_arrays_pointer_types}

C$\star$ is fairly cohesive with its syntactical provisions for arrays and
pointer types. \textbf{Laws} can be applied to all members of an array
individually using a star \texttt{\**}, or to specific members as desired.
Laws applied to array types also apply to pointer types in the situations they
are treated as arrays, i.e. used with the subscript operator \texttt{[]}.
In any other case, pointer types are treated as the numerics that they are,
as shown in previous sections. Observe:

\texttt{law foo : u16[] \\*
\{ \\*
\hspace*{1cm}\_[0] != 0; \\*
\hspace*{1cm}\_[\**] != 0xFFFF; \\*
\};}

Laws applied to specific members do not construe any information about the
size of an array, but merely apply if and when the specified element is dealt
with. If a law is applied to the $n^{th}$ element of an array, arrays must
have an $n^{th}$ element for that to be in effect.

Another powerful feature for array \textbf{laws} is that the index of the
array can also be qualified as an expression in much the same way the value of
it can. A bracket notation with no attached variable being subscripted is used
to denote this, like so:

\texttt{law bar : u16[] \\*
\{ \\*
\hspace*{1cm}\_[0] != 0; \\*
\hspace*{1cm}[\**] \% 2 == 0 \\*
\hspace*{1cm}\{ \\*
\hspace*{2cm}\_ != 0xFFFF; \\*
\hspace*{1cm}\}; \\*
\};}

This \textbf{law} asserts the following:

\begin{enumerate}
\item the first element of \texttt{u16[]}s cannot be zero; \textbf{and}
\item where the index of a \texttt{u16[]} is \textit{even}, the value therein
cannot equal \texttt{0xFFFF} (65,535).
\end{enumerate}

One thing that cannot be expressed plainly in \textbf{laws} is the notion of a
``valid pointer'', for example a pointer to valid memory returned from
\texttt{malloc()}. This is dealt with in the next section using \textit{data
tagging.}

\subsection{Tagging data in the AST}
\label{s-language-tagging_data_ast}

As is with \textbf{laws}, \textbf{tags} are information kept and acted upon at
compile time. \textbf{Tags} specifically are string literals kept in the AST
that can have certain meanings depending on their contents. This is useful for
communicating things that cannot be ascertained imperatively, providing an
alternative \textit{declarative} solution instead. One common case of this is
a function returning a pointer, and the caller needing that pointer to be
``valid'' according to some unspecified conditions only privy to the callee.
Observe:

\texttt{
void \** malloc( ptri sz ) \\*
\{ \\*
\hspace*{1cm}void \** mem \{ `alloc` \}; \\*
 \\*
\hspace*{1cm}/\** ... \**/ \\*
 \\*
\hspace*{1cm}if( err ) \\*
\hspace*{1cm}\{ \\*
\hspace*{2cm}return NULL; \\*
\hspace*{1cm}\} \\*
 \\*
\hspace*{1cm}return mem; \\*
\}}

In this function, the variable \texttt{mem} is decorated with a tag by the
name of \texttt{alloc}. Wherever this value flows through the call graph
within the total system, this tag will go with it, and be available for
assertions in the bodies of \textbf{laws}, where the brace format follows
the field, acting as a boolean expression like so:

\texttt{
struct foo \\*
\{ \\*
\hspace*{1cm}void \** obj; \\*
\hspace*{1cm}ptri sz; \\*
\}; \\*
 \\*
law : struct foo \\*
\{ \\*
\hspace*{1cm}\_.obj \{ `alloc` \}; \\*
\hspace*{1cm}\_.sz > 0; \\*
\};}

If the \texttt{.obj} member of any \texttt{struct foo} was not tagged as
allocated like shown above and is read anywhere, the law will be violated
and compilation will halt due to illegal code.

Wherever the brace tagging form is applied, multiple tags can be applied or
evaluated at once using the comma operator in between each backticked tag.

\subsection{Jurisdiction and pacts}
\label{s-language-jurisdiction_pacts}

TBW

\subsection{Marshalling}
\label{s-language-marshalling}

TBW

\section{Lexing}
\label{s-lexing}

The permitted encodings of source text is left to the \textbf{implementation}.
This document refers to source text agnostically to this, in terms of Unicode
codepoints. More exactly, this document will refer to the text as a sequence
of \textbf{runes}.

The text is tokenised into whitespace, end-of-line tokens, comments, and
finally regular tokens. It is then terminated with a special end-of-file
token.

The text is split into tokens using a ``maximal munch'' algorithm; in other
words, the lexer attempts to create the longest possible token.

\subsection{Runes}
\label{s-lexing-runes_characters}

\texttt{\textit{Rune} := \textit{a single Unicode codepoint}}

\texttt{\textit{Character} := \underline{\textit{PrintableChar}} |
\underline{\textit{NonPrintableChar}}}

\texttt{\textit{NonPrintableChar} := \textit{ASCII 0 .. 31} |
\textit{ASCII 127}}

\texttt{\textit{PrintableChar} : \textit{ASCII 32 .. 126}}

\subsection{End-of-file}
\label{s-lexing-endoffile}

\texttt{\textit{EndOfFile} := \textit{physical end of the file} |
\underline{\textbf{U+0000}} | \underline{\textbf{U+001A}}}

The source text is terminated by whichever of these appears first.

\subsection{End-of-line}
\label{s-lexing-endofline}

\texttt{\textit{EndOfLine} := \underline{\textbf{U+000D}}
\underline{\textbf{U+000A}} | \underline{\textbf{U+000D}} |
\underline{\textbf{U+000A}} | \underline{\textbf{U+2028}} |
\underline{\textbf{U+2029}} | \underline{\textit{EndOfFile}}}

\subsection{Whitespace}
\label{s-lexing-whitespace}

\texttt{\textit{Whitespace} := \underline{\textit{Space}} | \underline{\textit{Space}} \underline{\textit{Whitespace}}}

\texttt{\textit{Space} := \underline{\textbf{U+0020}} |
\underline{\textbf{U+0009}} | \underline{\textbf{U+000B}} |
\underline{\textbf{U+000C}}}

\subsection{Comments}
\label{s-lexing-comments}

\texttt{\textit{Comment} := \textbf{/\**} \underline{\textit{Runes}}
\textbf{\**/}}

\texttt{\textit{Runes} := \underline{\textit{Rune}}
\underline{\textit{Runes}} | \underline{\textit{Rune}}}

\subsection{Tokens}
\label{s-lexing-tokens}

\texttt{\textit{Token} := \underline{\textit{ASCIIStringLiteral}} |
\underline{\textit{UnicodeStringLiteral}} | \underline{\textit{CharLiteral}} |
\\* \underline{\textit{RuneLiteral}} | \underline{\textit{NumericLiteral}} |
\underline{\textit{Keyword}} | \underline{\textit{BuiltinType}} |
\underline{\textit{TypeModifier}} | \\* \underline{\textit{Identifier}} |
\textbf{>>>=} | \textbf{>>=} | \textbf{<<=} | \textbf{||=} | \textbf{|=} |
\textbf{\&\&=} | \textbf{\&=} | \textbf{\**=} |
\textbf{\textasciicircum\textasciicircum=} | \textbf{\textasciicircum=} |
\\* \textbf{+=} | \textbf{-=} | \textbf{/=} | \textbf{\%=} |
\textbf{\textasciitilde=} | \textbf{<=} | \textbf{>=} | \textbf{!=} |
\textbf{==} | \textbf{=} | \textbf{>>>} | \textbf{>>} | \textbf{<<} |
\textbf{||} | \textbf{|} | \\* \textbf{\&\&} | \textbf{\&} | \textbf{\**} |
\textbf{\textasciicircum\textasciicircum} | \textbf{\textasciicircum} |
\textbf{++} | \textbf{+} | \textbf{--} | \textbf{-} | \textbf{/} |
\textbf{\%} | \textbf{\textasciitilde} | \textbf{<} | \textbf{>} |
\textbf{!} | \textbf{...} | \textbf{.} | \\* \textbf{?} | \textbf{,} |
\textbf{;} | \textbf{:} | \textbf{(} | \textbf{)} | \textbf{[} | \textbf{]} |
\textbf{\{} | \textbf{\}}}

\subsection{Identifiers}
\label{s-lexing-identifiers}

\texttt{\textit{Identifier} := \underline{\textit{IdentifierStart}}
\underline{\textit{IdentifierRunes}} | \underline{\textit{IdentifierStart}}}

\texttt{\textit{IdentifierRunes} := \underline{\textit{IdentifierRune}}
\underline{\textit{IdentifierRunes}} | \underline{\textit{IdentifierRune}}}

\texttt{\textit{IdentifierStart} := \textbf{\_} | \textbf{A} | \textbf{B} |
\textbf{C} | \textbf{D} | \textbf{E} | \textbf{F} | \textbf{G} | \textbf{H} |
\textbf{I} | \textbf{J} | \textbf{K} | \textbf{L} | \\* \textbf{M} |
\textbf{N} | \textbf{O} | \textbf{P} | \textbf{Q} | \textbf{R} | \textbf{S} | \textbf{T} | \textbf{U} | \textbf{V} | \textbf{W} | \textbf{X} | \textbf{Y} | \textbf{Z} | \textbf{a} | \textbf{b} | \textbf{c} | \textbf{d} | \\* \textbf{e} | \textbf{f} | \textbf{g} | \textbf{h} | \textbf{i} | \textbf{j} | \textbf{k} | \textbf{l} | \textbf{m} | \textbf{n} | \textbf{o} | \textbf{p} | \textbf{q} | \textbf{r} | \textbf{s} | \textbf{t} | \textbf{u} | \textbf{v} | \\* \textbf{w} | \textbf{x} | \textbf{y} | \textbf{z}}

\texttt{\textit{IdentifierRune} := \underline{\textit{IdentifierStart}} |
\textbf{0} | \textbf{1} | \textbf{2} | \textbf{3} | \textbf{4} | \textbf{5} |
\textbf{6} | \textbf{7} | \textbf{8} | \textbf{9}}

\subsection{String literals}
\label{s-lexing-string_literals}

\texttt{\textit{ASCIIStringLiteral} := \textbf{"} \underline{\textit{StringChars}}
\textbf{"}}

\texttt{\textit{StringChars} := \underline{\textit{StringChar}}
\underline{\textit{StringChars}} | \underline{\textit{StringChar}}}

\texttt{\textit{StringChar} := \underline{\textit{ASCIIEscapeSequence}} |
\underline{\textit{PrintableChar}}}

\texttt{\textit{ASCIIEscapeSequence} :=
\textbf{\textbackslash\textbackslash} | \textbf{\textbackslash{}0} |
\textbf{\textbackslash{}a} | \textbf{\textbackslash{}b} |
\textbf{\textbackslash{}f} | \textbf{\textbackslash{}n} |
\textbf{\textbackslash{}r} | \textbf{\textbackslash{}t} |
\textbf{\textbackslash{}v} | \\* \textbf{\textbackslash}
\underline{\textit{OctDigit}} \underline{\textit{OctDigit}}
\underline{\textit{OctDigit}}}

\texttt{\textit{UnicodeStringLiteral} := \textbf{@"}
\underline{\textit{StringRunes}} \textbf{"}}

\texttt{\textit{StringRunes} := \underline{\textit{StringRune}}
\underline{\textit{StringRunes}} | \underline{\textit{StringRune}}}

\texttt{\textit{StringRune} := \underline{\textit{UnicodeEscapeSequence}} |
\underline{\textit{Rune}}}

\texttt{\textit{UnicodeEscapeSequence} :=
\underline{\textit{ASCIIEscapeSequence}} | \\* \textbf{\textbackslash{}u}
\underline{\textit{HexDigit}} \underline{\textit{HexDigit}}
\underline{\textit{HexDigit}} \underline{\textit{HexDigit}} | \\*
\textbf{\textbackslash{}U} \underline{\textit{HexDigit}}
\underline{\textit{HexDigit}} \underline{\textit{HexDigit}}
\underline{\textit{HexDigit}} \underline{\textit{HexDigit}}
\underline{\textit{HexDigit}} \underline{\textit{HexDigit}}
\underline{\textit{HexDigit}}}

\subsection{Rune and character literals}
\label{s-lexing-rune_character_literals}

\texttt{\textit{CharLiteral} := \textbf{'}
\underline{\textit{SingleQuotedChar}} \textbf{'}}

\texttt{\textit{SingleQuotedRune} := \underline{\textit{ASCIIEscapeSequence}}
| \underline{\textit{PrintableChar}}}

\texttt{\textit{RuneLiteral} := \textbf{@'}
\underline{\textit{SingleQuotedRune}} \textbf{'}}

\texttt{\textit{SingleQuotedRune} :=
\underline{\textit{UnicodeEscapeSequence}} | \underline{\textit{Rune}}}

\subsection{Numeric literals}
\label{s-lexing-numeric_literals}

\texttt{\textit{NumericLiteral} := \underline{\textit{IntegerLiteral}} |
\underline{\textit{FloatLiteral}}}

\texttt{\textit{IntegerLiteral} := \underline{\textit{Integer}}
\underline{\textit{IntegerSuffix}} | \underline{\textit{Integer}}}

\texttt{\textit{Integer} := \underline{\textit{OctInteger}} |
\underline{\textit{DecInteger}} | \underline{\textit{HexInteger}}}

\texttt{\textit{IntegerSuffix} := \textbf{u} | \textbf{U} | \textbf{l} |
\textbf{L} | \textbf{ul} | \textbf{Ul} | \textbf{uL} | \textbf{UL} |
\textbf{lu} | \textbf{Lu} | \textbf{lU} | \textbf{LU}}

\texttt{\textit{OctInteger} := \underline{\textit{OctDigit}}
\underline{\textit{OctInteger}} | \underline{\textit{OctDigit}}}

\texttt{\textit{OctDigit} := \textbf{0} | \textbf{1} | \textbf{2} |
\textbf{3} | \textbf{4} | \textbf{5} | \textbf{6} | \textbf{7}}

\texttt{\textit{DecInteger} := \underline{\textit{NonZeroDigit}}
\underline{\textit{DecDigits}} | \underline{\textit{NonZeroDigit}}}

\texttt{\textit{NonZeroDigit} := \textbf{1} | \textbf{2} | \textbf{3} |
\textbf{4} | \textbf{5} | \textbf{6} | \textbf{7} | \textbf{8} | \textbf{9}}

\texttt{\textit{DecDigits} := \underline{\textit{DecDigit}}
\underline{\textit{DecDigits}} | \underline{\textit{DecDigit}}}

\texttt{\textit{DecDigit} := \textbf{0} | \underline{\textit{NonZeroDigit}}}

\texttt{\textit{HexInteger} := \underline{\textit{HexPrefix}}
\underline{\textit{HexDigits}}}

\texttt{\textit{HexPrefix} := \textbf{0x} | \textbf{0X}}

\texttt{\textit{HexDigits} := \underline{\textit{HexDigit}}
\underline{\textit{HexDigits}} | \underline{\textit{HexDigit}}}

\texttt{\textit{HexDigit} := \underline{\textit{DecDigit}} |
\underline{\textit{HexLetters}}}

\texttt{\textit{HexLetters} := \textbf{A} | \textbf{B} | \textbf{C} |
\textbf{D} | \textbf{E} | \textbf{F} | \textbf{a} | \textbf{b} | \textbf{c} |
\textbf{d} | \textbf{e} | \textbf{f}}

\texttt{\textit{FloatLiteral} := \underline{\textit{Float}}
\underline{\textit{FloatSuffix}} \underline{\textit{ImaginarySuffix}} | \\*
\underline{\textit{Float}} \underline{\textit{RealSuffix}}
\underline{\textit{ImaginarySuffix}} | \underline{\textit{Float}}
\underline{\textit{FloatSuffix}} | \underline{\textit{Float}}
\underline{\textit{RealSuffix}} | \\* \underline{\textit{Float}}
\underline{\textit{ImaginarySuffix}} | \underline{\textit{Float}}}

\texttt{\textit{Float} := \underline{\textit{DecFloat}} |
\underline{\textit{HexFloat}}}

\texttt{\textit{DecFloat} := \underline{\textit{DecDigits}} \textbf{.}
\underline{\textit{DecDigits}} \underline{\textit{Exponent}} | 
\underline{\textit{DecDigits}} \textbf{.} \underline{\textit{DecDigits}} | \\*
\textbf{.} \underline{\textit{DecDigits}} \underline{\textit{DecExponent}} |
\textbf{.} \underline{\textit{DecDigits}} | \underline{\textit{DecDigits}}
\underline{\textit{DecExponent}} | \underline{\textit{DecDigits}} \textbf{.}}

\texttt{\textit{DecExponent} := \underline{\textit{DecExponentStart}}
\underline{\textit{DecDigits}}}

\texttt{\textit{DecExponentStart} := \textbf{e+} | \textbf{E+} | \textbf{e-} |
\textbf{E-} | \textbf{e} | \textbf{E}}

\texttt{\textit{HexFloat} := \underline{\textit{HexPrefix}}
\underline{\textit{HexDigits}} \textbf{.} \underline{\textit{HexDigits}}
\underline{\textit{HexExponent}} | \\* \underline{\textit{HexPrefix}}
\textbf{.} \underline{\textit{HexDigits}} \underline{\textit{HexExponent}} |
\underline{\textit{HexPrefix}} \underline{\textit{HexDigits}}
\underline{\textit{HexExponent}}}

\texttt{\textit{HexExponent} := \underline{\textit{HexExponentStart}}
\underline{\textit{DecDigits}}}

\texttt{\textit{HexExponentStart} := \textbf{p+} | \textbf{P+} | \textbf{p-} |
\textbf{P-} | \textbf{p} | \textbf{P}}

\texttt{\textit{FloatSuffix} := \textbf{f} | \textbf{F}}

\texttt{\textit{RealSuffix} := \textbf{l} | \textbf{L}}

\texttt{\textit{ImaginarySuffix} := \textbf{i} | \textbf{I}}

\subsection{Keywords}
\label{s-lexing-keywords}

Keywords are reserved identifiers. They fall into one of two categories:
declarators and control flow. The keywords \texttt{enum},
\texttt{law}, \texttt{noalloc}, \texttt{nothrow}, \texttt{noreturn},
\texttt{pure}, \texttt{struct} and \texttt{union} keywords are all of the
declarator keywords, and the remaining ones are for control flow.

\texttt{\textit{Keyword} := \textbf{areaof} | \textbf{break} | \textbf{case} |
\textbf{continue} | \textbf{default} | \textbf{do} | \textbf{else} |
\textbf{enum} | \\* \textbf{for} | \textbf{goto} |
\textbf{if} | \textbf{law} | \textbf{marshal} | \textbf{noalloc} |
\textbf{nothrow} | \textbf{noreturn} | \textbf{pure} | \\*
\textbf{return} | \textbf{sizeof} | \textbf{struct} | \textbf{switch} |
\textbf{typedef} | \textbf{union} | \textbf{while}}

\subsection{Builtin Types}
\label{s-lexing-builtin_types}

C$\star$ provides sixteen (16) primitive base types. The \texttt{void} type
is an incomplete type that cannot be completed. The \texttt{char} type is used
for holding text or other 8-bit character data; its signedness is
\textbf{implementation}-defined. The \texttt{rune} type is used for holding
Unicode code points, being at least 21 bits wide. The \texttt{ptri} type is an
integral sized the same to the \textbf{machine}'s pointers; it is unsigned.
The \texttt{offs} type is defined the same as \texttt{ptri} but it is signed,
as it is intended for use with offsets. The \texttt{s\{8,16,32,64\}} are
signed integers 8, 16, 32 and 64 bits wide, respectively. The
\texttt{u\{8,16,32,64\}} are unsigned integers 8, 16, 32 and 64 bits wide,
respectively. The \texttt{f\{16,32,64\}} are IEEE 754 compliant binary
floating-point numbers with half, single and double precision, respectively.

\texttt{\textit{BuiltinType} := \textbf{char} | \textbf{f16} | \textbf{f32} |
\textbf{f64} | \textbf{offs} | \textbf{ptri} | \textbf{s8} | \textbf{s16} |
\textbf{s32} | \\* \textbf{s64} | \textbf{rune} | \textbf{u8} | \textbf{u16} |
\textbf{u32} | \textbf{u64} | \textbf{void}}

\subsection{Storage modifiers}
\label{s-lexing-storage_modifiers}

Storage modifiers are a kind of keywords that change the meaning of a
variable declaration’s storage class. They precede the typing of a variable
in its declaration. Declaring a variable with \texttt{const} causes it to be
construed as belonging in \textbf{constant memory}. Declaring a variable with
\texttt{volatile} causes it to be construed as residing in \textbf{shared
memory}, and also construes it as being modifiable from outside the
\textbf{program}. Declaring a variable with \texttt{extern} construes it as a
``forward declaration'', meaning the true location of the variable is defined
outside the \textbf{program}.

\texttt{\textit{StorageModifier} := \textbf{const} | \textbf{extern} |
\textbf{register} | \textbf{volatile}}

\clearpage{}

\end{document}
